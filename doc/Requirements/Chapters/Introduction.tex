% !TeX spellcheck = en_GB
\chapter{Introduction}
The product requirements document (PRD) contains all requirements that are mandatory on the system to be developed. 
It is the basis for tendering and thus the most important specification for tender preparation. 
The requirements define the framework conditions for development, which are then detailed by the group "Programming" in the functional specification document (FSD).
All relevant requirements for the system are determined and documented by the "Analysis" group. 
They contain the information necessary for the group "Programming" to develop the required system. 
The core of the product requirements document are the functional and non-functional requirements of the system, as well as a sketch of the overall system design. 
The draft takes into account the future environment and infrastructure in which the system will operate later.
The functional and non-functional requirements serve not only as guidelines for the development but are also the basis for the requirements traceability and the change measures. 
The requirements should be prepared so that the traceability, as well as appropriate change measures for the entire life cycle of a system is possible. 
In general, the product requirement document should not specify any technical solutions for not restrict architects and developers the search of optimal technical solutions.

