% !TeX spellcheck = en_GB
\chapter{Main Objectives}
\section{Overview}
\section{Wiki}
\section{Archive}
\section{User types}
\section{BSI Catalog}
\section{Search function}
\label{search_function}

\section{Wizard}
\label{wizard}

\section{Frontpage}
The frontpage is the entry point to the different services offered by the website. 
As such its design should provide a clear overview of and a dead on target guidance to all available functions for users of all levels of experience.
Firstly, the frontpage also displays the always present top section (see section~\ref{top_section}) but no side bar.
Directly below the top section is a distinguished area for time-critical news which inform of widespread threats or important updates.
In times of no imminent danger time-critical news might not be displayed but instead a few of the most recent regular news which are part of the wiki.
The third area in a vertical sense is a wide and inviting search bar that allows experienced users the quick access to the BSI catalogue and other parts. 
The search offers the full functionality as described in section~\ref{search_function}.
As a last section before the always present bottom section is an overview of introductory tutorials on how to implement the guidelines of the BSI catalgue while developing, building and maintaining a basic IT system.
Equally visible as the tutorials should be the offer to use the wizard to help and find security gaps and other system flaws.
Both the tutorials and the wizard are aimed at users of no or little knowledge or overview of the BSI catalogue.
For detailed explanations see section~\ref{tutorials} for the tutorials and section~\ref{wizard} for the wizard.

\section{Top Section}
\label{top_section}

The top section is an always present area at the top of each subpage that connects the different services and allows for quick access.
It should feature the following items whose order and wording might be changed appropriately:
\begin{itemize}
    \item Home/Frontpage
    \item News
    \item Browse the BSI catalogue
    \item Tutorials
    \item Wizard
    \item Login
        .
\end{itemize}

\section{Tutorials}
\label{tutorials}

Tutorials should call unexperienced users' attention to the most important points in the BSI catalogue when developing, building or maintaining an IT infrastructure.
They could briefly explain mayor points of the BSI catalogue and indicate next steps.
The simplest tutorial could simply introduce the usage and goal of the website and its subservices.
The tutorials are not meant to be a rewrite - i.e. the BSI catalogue for dummies - but thought of as a quick overview and guiding introduction into the matter.
They are part of the wiki and as such created and maintained by content manager and linked by mods.
