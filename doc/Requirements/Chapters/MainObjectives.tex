% !TeX spellcheck = en_GB
\chapter{Main Objectives}
\section{Overview}
The goal of this project is to design, implement and provide a guidance Platform for development Country, e.g. Afghanistan. Taking the circumstances in such development Country in account, the platform should provide the users of IT-Systems with the knowledge to assure the sustainability of IT-Systems in their countries. Besides providing general information of the IT-Systems, the platform should provide, similar as in the “BSI Grundschutzkatalog”, information aiming to localize and solve problematics that may occur to the IT-Systems as a result of a human error, technical failure or a catastrophe etc. 

\section{Wiki}
Considering the limited availability of IT-specialists in most of the development countries, and after reviewing the possible ways to design and provide the platform, our team consider it convenient to create an open Platform in which people can collaboratively add, edit, delete or archive content (Wiki). This should allow the platform to constantly grow and stay relevant to the current circumstances. This anyway shouldn’t mean that everyone can write whatever content for everybody to read. To assure that we have to set rules and access permission levels. 

\section{Archive}
\section{User types}
\section{BSI Catalog}
The English version of the BSI catalogue has been published in a single PDF file. This makes browsing, or even searching a specific problem, a difficult task for beginners. Therefore the BSI Catalogue should be added as an "locked" Article, only modifiable by Administrators.
\\\\
In fact, converting catalogues in articles by hand would be an interminable task. Accordingly, a parser is needed. This parser will analyse the PDF file and create the article automatically. Further, the BSI provides cross-reference tables of the catalogue. These amend the useful links inside each subsection and could be used to recommend cross-referenced counter-measures to specific threats.
\section{Text search}
A text search function allows users of the platform a possibility to make a free text search request and browse the results containing the key-words. Users should also have the opportunity to narrow their search request by selecting a specific topic in which they would like to find results. These topic are: "All", "News", "Module", "Threats", "Counter-Measures", "Archive". Choices should be available as a dropdown list.
\\\\
The search field should be well placed on the home page. The system should have a function "Back to the search results". The search function should have fault tolerance, as well as ability to deal with synonyms. Fault tolerance means that the user will get the result even if he misspells a word or if he uses singular or plural. While a non-fault-tolerant search will let the visitor go nowhere, the system should nevertheless display the appropriate results.
In addition Search functions, which also master synonyms, do even more. For example, the system should derive results from the search input "laptop" in which the word "notebook" appears and for this purpose access an extensive database of words of the same meaning. The automatic completion of search terms is very welcome.
\section{Wizzard}
Beginners who are not familiar with the terminology may have a hard time finding solutions to problems they encounter. To help them, we would like to implement a "Wizzard", which will ask them a set of yes/no questions to filter out what problems they could have, similar to the game Akinator\footnote{\url{http://en.akinator.com}}. 
\section{Frontpage}
\section{Tutorials}